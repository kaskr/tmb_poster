\documentclass[largefonts]{sciposter}
\usepackage{Sweave}
% \DefineVerbatimEnvironment{Sinput}{Verbatim} {xleftmargin=2em,
%                                               frame=single}
% \DefineVerbatimEnvironment{Soutput}{Verbatim}{xleftmargin=2em,
%                                               frame=single}

\usepackage{listings}
%\usepackage{epsfig}
% \lstnewenvironment{code}[1][]%
%   {\minipage{\linewidth} 
%    \lstset{basicstyle=\ttfamily\footnotesize,frame=single,#1}}
%   {\endminipage}


\usepackage{amsmath}
\usepackage{amssymb}
\usepackage{multicol}
% \usepackage{namedplus}

%\usepackage{fancybullets}

\newcommand{\imsize}{0.45\columnwidth}

\newtheorem{Def}{Definition}
\newcommand{\mb}[1]{\mathbf{#1}}

%\setkeys{Gin}{width=0.3\textwidth}

%\definecolor{BoxCol}{rgb}{0.9,0.9,0.9}
% uncomment for grey background to \section boxes 
% for use with default option boxedsections

% \definecolor{mainCol}{rgb}{0.9,1,1}
\definecolor{BoxCol}{rgb}{0.9,0.9,1}
% uncomment for light blue background to \section boxes 
% for use with default option boxedsections

\definecolor{SectionCol}{rgb}{0,0,0.5}
% uncomment for dark blue \section text 

%% ==================== lst colors
%\lstset{numbers=left, stepnumber=2, frame=single,}
%\lstset{frame=shadowbox, rulesepcolor=\color{blue}, backgroundcolor=\color{white},basicstyle=\footnotesize}
\definecolor{mygreen}{rgb}{0,0.6,0}
\definecolor{mygray}{rgb}{0.5,0.5,0.5}
\definecolor{mymauve}{rgb}{0.58,0,0.82}
\lstset{ %
  backgroundcolor=\color{white},   % choose the background color; you must add \usepackage{color} or \usepackage{xcolor}
  basicstyle=\footnotesize,        % the size of the fonts that are used for the code
  breakatwhitespace=false,         % sets if automatic breaks should only happen at whitespace
  breaklines=true,                 % sets automatic line breaking
  captionpos=t,                    % sets the caption-position to bottom
  commentstyle=\color{mygreen},    % comment style
  deletekeywords={...},            % if you want to delete keywords from the given language
  escapeinside={\%*}{*)},          % if you want to add LaTeX within your code
  extendedchars=true,              % lets you use non-ASCII characters; for 8-bits encodings only, does not work with UTF-8
  frame=single,                    % adds a frame around the code
  keepspaces=true,                 % keeps spaces in text, useful for keeping indentation of code (possibly needs columns=flexible)
  keywordstyle=\color{blue},       % keyword style
  language=c++,                 % the language of the code
  morekeywords={*,...},            % if you want to add more keywords to the set
  numbers=left,                    % where to put the line-numbers; possible values are (none, left, right)
  numbersep=5pt,                   % how far the line-numbers are from the code
  numberstyle=\tiny\color{mygray}, % the style that is used for the line-numbers
  rulecolor=\color{black},         % if not set, the frame-color may be changed on line-breaks within not-black text (e.g. comments (green here))
  showspaces=false,                % show spaces everywhere adding particular underscores; it overrides 'showstringspaces'
  showstringspaces=false,          % underline spaces within strings only
  showtabs=false,                  % show tabs within strings adding particular underscores
  stepnumber=2,                    % the step between two line-numbers. If it's 1, each line will be numbered
  stringstyle=\color{mymauve},     % string literal style
  tabsize=2,                       % sets default tabsize to 2 spaces
  %title=\lstname                   % show the filename of files included with \lstinputlisting; also try caption instead of title
}



\title{Template Model Builder (TMB)}

% Note: only give author names, not institute
\author{Kasper Kristensen, Anders Nielsen, Casper Berg, Hans Skaug and Brad Bell}
 
% insert correct institute name
\institute{DTU Compute, DTU Aqua, University of Bergen, IHME UW}
\email{kaskr@imm.dtu.dk}  % shows author email address below institute

%\date is unused by the current \maketitle


% The following commands can be used to alter the default logo settings

%\leftlogo[0.8]{Uffe-Thygesen-001-snip}
%\leftlogo[0.8]{DTU_logo_large}
% defines logo to left of title (with scale factor)
\rightlogo[0.52]{DTU_logo_large}  % same but on right

% NOTE: This will require presence of files logoWenI.eps and RuGlogo.eps, 
% or other supported format in the current directory  
%%%%%%%%%%%%%%%%%%%%%%%%%%%%%%%%%%%%%%%%%%%%%%%%%%%%%%%%%%%%%%%%%%%%%%%%%%%%%%%%
%%% Begin of Document



\begin{document}
%define conference poster is presented at (appears as footer)

\conference{{\bf DSTS November 2013}}

%\LEFTSIDEfootlogo  
% Uncomment to put footer logo on left side, and 
% conference name on right side of footer

% Some examples of caption control (remove % to check result)

%\renewcommand{\algorithmname}{Algoritme} % for Dutch

%\renewcommand{\mastercapstartstyle}[1]{\textit{\textbf{#1}}}
%\renewcommand{\algcapstartstyle}[1]{\textsc{\textbf{#1}}}
%\renewcommand{\algcapbodystyle}{\bfseries}
%\renewcommand{\thealgorithm}{\Roman{algorithm}}

\maketitle

%%% Begin of Multicols-Enviroment
\begin{multicols}{3}

%%% Abstract

%\begin{abstract}

\section*{Summary}
TMB is a tool, that allows a user to quickly implement complex random
effect models through simple c++ templates. It is implemented as an
open source R package with functionality similar to the established tool AD Model Builder
(admb-project.org). The user implements the model in terms of the
joint distribution of random and fixed effects. Given such an
implementation -- here referred to as a \emph{template} -- TMB carries
out a Laplace approximation of the marginal likelihood function with
exact derivatives obtained through automatic differentiation. TMB is
designed be to fast for problems with many random effects
($\approx 10^6$) and fixed effects ($\approx 10^3$).

\section{Key features}
\begin{itemize}
  \item Combines established high performance libraries: CppAD, Eigen, CHOLMOD
  \item Implements Laplace approximation for random effects with exact derivatives.
  \item C++ Template based with R-like syntax
  \item Automatic sparseness detection
  \item Parallelism through BLAS
  \item Parallel user templates
  \item Parallelism through R's \texttt{multicore} package 
\end{itemize}

\section{Applications}
Problems that can be solved using TMB include
\begin{itemize}
\item Non-linear mixed effect modelling
\item State-space modelling
\item Linear or non-linear filtering
\item Missing data problems
\item Time series analysis
\item Space time modelling

\item SDEs and SPDEs
\item Big data
\end{itemize}
and more.


%\textbf{We show examples of templates for three selected models.}



\section{Example: Linear regression}
\input{linreg.tex}

\section{Example: Multivariate state space model}
\input{rw.tex}

\section{Conclusion}
TMB is a highly flexible and efficient tool for statisticians working
with non-standard model development. This poster has only demonstrated
close to trivial applications: The state-space model with 300 states
took 0.3 seconds to run including maximum likelihood estimation of
fixed effects and prediction of the latent state variables; A wide
range of examples are available on the project hosting site (www.github.com/kaskr/adcomp).\\
TMB has been tested to work on common platforms: Linux, Mac OSX and
Windows.

\end{multicols}
\end{document}

